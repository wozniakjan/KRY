
\documentclass[a4paper,11pt]{article}
\usepackage[czech]{babel}
\usepackage[utf8]{inputenc}
%\usepackage[IL2]{fontenc}
\usepackage[left=1.5cm,text={18cm, 25cm},top=2.5cm]{geometry}
\usepackage{color}
\usepackage[unicode, colorlinks,hyperindex,plainpages=false,pdftex]{hyperref}
\usepackage{graphicx}


\begin{document}

\begin{center}{\LARGE\textbf{Implementace a prolomení RSA}}\\[0.2cm]
\newcommand{\autor}[2]{#1&\texttt{#2@stud.fit.vutbr.cz}\tabularnewline}
\begin{tabular}{ll}
    \autor{Jan Wozniak}{xwozni00}
\end{tabular}
\end{center}


\section{Úvod}
Tato dokumentace se zabývá popisem algoritmů a použitých knihoven pro splnění druhého projektu
do předmětu KRY -- implementace a prolomení RSA. Úkolem bylo vytvořit program který pomocí
parametrů příkazové řádky umožní uživateli generovat parametry RSA (veřejného a soukromého
klíče), šifrovat, dešifrovat a prolomit RSA pomoci faktorizace slabého veřejného modulu, čímž
je demonstrována důležitost použití dostatečně dlouhých klíčů.

\section{RSA}
Algoritmus RSA spadá mezi asymetrické šifry. Je založen na předpokladu, že faktorizace --
rozklad čísla na součín prvočísel, pro dostatečně velkou hodnotu je obtížná úloha.
\subsection*{Generování parametrů}
Tato sekce se skládá z několika kroků:
\begin{itemize}
\item generuj dvě velká prvočísla $p$ a $q$
\item $n = p \cdot q$
\item $\phi(n) = (p-1) \cdot (q-1)$
\item $e = 3$
\item vypočti $d = inv(e, \phi(n))$
\item veřejný klíč je dvojice $(e,n)$ a soukromý klíč je dvojice $(d,n)$
\end{itemize}

Pro generování čísel $p$ a $q$ je použita funkce \texttt{rand()} ze standardní knihovny, 
inicializováno seedem z náhodné hodnoty ze souboru \texttt{/dev/urand}. Generované byty
jsou zapsáný do bufferu o patřičné délce a ten je převeden na \texttt{mpz\_t} hodnotu.
Aby bylo zajištěno, že p a q jsou s dostatečnou pravděpodobností prvočísla, je implementován
Miller-Rabin test prvočíselnosti a při neúspěchu je hodnota inkrementována a opětovně testována.
Modulární multiplikativní inverze je spočtena pomocí rozšířeného Eukleidova algoritmu.

\subsection*{Šifrování a dešifrování}
Jelikož na vstupu je očekávána konkrétní číselná hodnota pro šifrování a dešifrování,
není implementován žádný algoritmus pro zpracování vstupního řetězce jako např. PKCS \#1 v1.5.
Vstupní hodnota je jen vynásobena
\begin{itemize}
\item šifrování $c = m^e (mod n)$
\item dešifrování $m = c^d (mod n)$
\end{itemize}

Modulární násobení je implementováno pomocí algoritmu \textit{Montgomery Powering Ladder}.

\section{Prolomení slabého veřejného modulu}

\section{Závěr}


\end{document}
