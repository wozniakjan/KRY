
\documentclass[a4paper,11pt]{article}
\usepackage[czech]{babel}
\usepackage[utf8]{inputenc}
%\usepackage[IL2]{fontenc}
\usepackage[left=1.5cm,text={18cm, 25cm},top=2.5cm]{geometry}
\usepackage{color}
\usepackage[unicode, colorlinks,hyperindex,plainpages=false,pdftex]{hyperref}
\usepackage{graphicx}


\begin{document}

\begin{center}{\LARGE\textbf{Implementace a prolomení RSA}}\\[0.2cm]
\newcommand{\autor}[2]{#1&\texttt{#2@stud.fit.vutbr.cz}\tabularnewline}
\begin{tabular}{ll}
    \autor{Jan Wozniak}{xwozni00}
\end{tabular}
\end{center}


\section{Úvod}
Tato dokumentace se zabývá popisem algoritmů a použitých knihoven pro splnění druhého projektu
do předmětu KRY -- implementace a prolomení RSA. Úkolem bylo vytvořit program který pomocí
parametrů příkazové řádky umožní uživateli generovat parametry RSA (veřejného a soukromého
klíče), šifrovat, dešifrovat a prolomit RSA pomoci faktorizace slabého veřejného modulu, čímž
je demonstrována důležitost použití dostatečně dlouhých klíčů.

\section{RSA}
Algoritmus RSA spadá mezi asymetrické šifry. Je založen na předpokladu, že faktorizace --
rozklad čísla na součín prvočísel, pro dostatečně velkou hodnotu je obtížná úloha.:
\subsection*{Generování parametrů}
\subsection*{Šifrování a dešifrování}

\section{Prolomení slabého veřejného modulu}

\section{Závěr}


\end{document}
