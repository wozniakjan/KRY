
\documentclass[a4paper,11pt]{article}
\usepackage[czech]{babel}
\usepackage[utf8]{inputenc}
%\usepackage[IL2]{fontenc}
\usepackage[left=1.5cm,text={18cm, 25cm},top=2.5cm]{geometry}
\usepackage{color}
\usepackage[unicode, colorlinks,hyperindex,plainpages=false,pdftex]{hyperref}
\usepackage{graphicx}


\begin{document}

\begin{center}{\LARGE\textbf{Vigenerova šifra}}\\[0.2cm]
\newcommand{\autor}[2]{#1&\texttt{#2@stud.fit.vutbr.cz}\tabularnewline}
\begin{tabular}{ll}
    \autor{Jan Wozniak}{xwozni00}
\end{tabular}
\end{center}


\section{Úvod}

Tato dokumentace stručně popisuje implementaci 1. projektu do předmětu kryptografie, prolomení hesla Vigenerovy šifry.
Program je napsaný v jazyce C/C++, na standardním vstupu očekává zašifrovaný anglický text a na standardní výstup vypíše
\texttt{[friedman\_test];[kaisisky\_test];[delka\_hesla];[heslo]} kde \texttt{friedman\_test} je výsledek Friedmanova
testu jako float čislo, \texttt{kaisisky\_test} je výsledek Kaisiského testu jako integer,
\texttt{delka\_hesla} je odhadnutá délka hesla na základě předchozích testů, \texttt{heslo} je odhadované heslo použité pro
zašifrování textu.

\section{Implementace}
Projekt je v jediném zdrojovém souboru \texttt{main.cpp} v adresáři \texttt{src/}. Pro urychlení operací s textem, zejména
vyhledávání shodných podřetězců v Kaisiského textu je využita pointerová aritmetika a zdrojový text je uložen v paměti jako
pole \texttt{char}. V rámci odstranění všech znaků mimo písmen je text rovněž převeden na lower-case.

Funkce \texttt{kaisisky()} očekává jediný parametr, a to profiltrovaný text, a vrací hodnotu tohoto testu. Ta volá nejprve
funkci \texttt{get\_frequencies()}, která tyhledává v textu shody na trigrafy, 4-grafy a 5-grafy. Poté volá funkci \texttt{get\_gcd()}
které mezi nalezenýmí frekvencemi vyhledá vhodného největšího společného dělitele a tím určí odhad délky hesla.

Funkce \texttt{friedman()} na základě frekvencí výskytu písmen v šifrovaném textu a principu \textit{indexu koincidence} 
spočte hodnotu friedmanova testu.

Posledním úkolem je zjištění hesla z textu, který se dá řešit díky známé délce hesla $n$ jako monoalfabetická šifra, neboť
text můžeme rozdělit na $n$ podřetězců z nichž každý podřetězec je právě jedna Caesarova šifra. Poté na základě frekvencí
výskytu písmen v textu jsme pomocí shiftování abecedy schopni určit, kdy je text nejpodobnější anglickému textu. Pro zjištění hesla
slouží funkce \texttt{set\_password()}.

Délka odhadu hesla je nakonec stanovena na výsledek Kaisiského testu, neboť po krátkém prozkoumání pár testovacích výsledků
byla tato hodnota vcelku přesná a reprezentovala docela věrně délku skutečného hesla, pokud to nebylo příliš dlouhé.

\section{Závěr}
V rámci tohoto projektu bylo za úkol odhadnout délku hesla a heslo samontné textu, který je šifrován Vigenerovou šifrou.
Implementované metriky jsou použitelné dobře pro delší anglické texty, neboť hodnoty jako frekvence výskytu písmen a 
vyhledávání shodných podřetězců mají vyšší informační hodnotu pro delší text a hodnoty konstant kappa jsou závislé na
jazyku textu.


\end{document}
